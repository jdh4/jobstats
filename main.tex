%%
%% This is file `sample-acmsmall-conf.tex',
%% generated with the docstrip utility.
%%
%% The original source files were:
%%
%% samples.dtx  (with options: `acmsmall-conf')
%% 
%% IMPORTANT NOTICE:
%% 
%% For the copyright see the source file.
%% 
%% Any modified versions of this file must be renamed
%% with new filenames distinct from sample-acmsmall-conf.tex.
%% 
%% For distribution of the original source see the terms
%% for copying and modification in the file samples.dtx.
%% 
%% This generated file may be distributed as long as the
%% original source files, as listed above, are part of the
%% same distribution. (The sources need not necessarily be
%% in the same archive or directory.)
%%
%% Commands for TeXCount
%TC:macro \cite [option:text,text]
%TC:macro \citep [option:text,text]
%TC:macro \citet [option:text,text]
%TC:envir table 0 1
%TC:envir table* 0 1
%TC:envir tabular [ignore] word
%TC:envir displaymath 0 word
%TC:envir math 0 word
%TC:envir comment 0 0
%%
%%
%% The first command in your LaTeX source must be the \documentclass command.
\documentclass[manuscript]{acmart}
%% NOTE that a single column version is required for 
%% submission and peer review. This can be done by changing
%% the \doucmentclass[...]{acmart} in this template to 
%% \documentclass[acmsmall,screen,review]{acmart}
%% Or use the sample-acmsmall-submission.tex file.
%% 
%% To ensure 100% compatibility, please check the white list of
%% approved LaTeX packages to be used with the Master Article Template at
%% https://www.acm.org/publications/taps/whitelist-of-latex-packages 
%% before creating your document. The white list page provides 
%% information on how to submit additional LaTeX packages for 
%% review and adoption.
%% Fonts used in the template cannot be substituted; margin 
%% adjustments are not allowed.

%%
%% \BibTeX command to typeset BibTeX logo in the docs
\AtBeginDocument{%
  \providecommand\BibTeX{{%
    \normalfont B\kern-0.5em{\scshape i\kern-0.25em b}\kern-0.8em\TeX}}}

%% Rights management information.  This information is sent to you
%% when you complete the rights form.  These commands have SAMPLE
%% values in them; it is your responsibility as an author to replace
%% the commands and values with those provided to you when you
%% complete the rights form.
\setcopyright{acmcopyright}
\copyrightyear{2023}
\acmYear{2023}
\acmDOI{XXXXXXX.XXXXXXX}

\acmConference[PEARC2023]{Practice and Experience in Advanced Research Computing}{June 23--27,
  2023}{Portland, OR}
%
%  Uncomment \acmBooktitle if th title of the proceedings is different
%  from ``Proceedings of ...''!
%
%\acmBooktitle{Woodstock '18: ACM Symposium on Neural Gaze Detection,
%  June 03--05, 2018, Woodstock, NY} 
\acmPrice{15.00}
\acmISBN{978-1-4503-XXXX-X/18/06}


%%
%% Submission ID.
%% Use this when submitting an article to a sponsored event. You'll
%% receive a unique submission ID from the organizers
%% of the event, and this ID should be used as the parameter to this command.
%%\acmSubmissionID{123-A56-BU3}

%%
%% For managing citations, it is recommended to use bibliography
%% files in BibTeX format.
%%
%% You can then either use BibTeX with the ACM-Reference-Format style,
%% or BibLaTeX with the acmnumeric or acmauthoryear sytles, that include
%% support for advanced citation of software artefact from the
%% biblatex-software package, also separately available on CTAN.
%%
%% Look at the sample-*-biblatex.tex files for templates showcasing
%% the biblatex styles.
%%

%%
%% The majority of ACM publications use numbered citations and
%% references.  The command \citestyle{authoryear} switches to the
%% "author year" style.
%%
%% If you are preparing content for an event
%% sponsored by ACM SIGGRAPH, you must use the "author year" style of
%% citations and references.
%% Uncommenting
%% the next command will enable that style.
%%\citestyle{acmauthoryear}

%%
%% end of the preamble, start of the body of the document source.
\begin{document}

%%
%% The "title" command has an optional parameter,
%% allowing the author to define a "short title" to be used in page headers.
\title[Jobstats: A Platform for Job Monitoring]{Jobstats: A Slurm-Compatible Job Monitoring Platform for CPU and GPU Clusters}

%%
%% The "author" command and its associated commands are used to define
%% the authors and their affiliations.
%% Of note is the shared affiliation of the first two authors, and the
%% "authornote" and "authornotemark" commands
%% used to denote shared contribution to the research.

\author{Josko Plazonic}
\email{plazonic@princeton.edu}
\affiliation{%
  \institution{OIT Research Computing, Princeton University}
  \city{Princeton}
  \state{New Jersey}
  \country{USA}
  \postcode{08544}
}

\author{Jonathan D. Halverson}
\email{halverson@princeton.edu}
\affiliation{%
  \institution{Princeton Institute for Computational Science and Engineering, Princeton University}
  \city{Princeton}
  \state{New Jersey}
  \country{USA}
  \postcode{08544}
}

\author{Troy J. Comi}
\email{tcomi@princeton.edu}
\affiliation{%
  \institution{OIT Research Computing \& Department of Chemical and Biological Engineering, Princeton University}
  \city{Princeton}
  \state{New Jersey}
  \country{USA}
  \postcode{08544}
}

%%
%% By default, the full list of authors will be used in the page
%% headers. Often, this list is too long, and will overlap
%% other information printed in the page headers. This command allows
%% the author to define a more concise list
%% of authors' names for this purpose.
\renewcommand{\shortauthors}{Plazonic, Halverson and Comi}

%%
%% The abstract is a short summary of the work to be presented in the
%% article.
%           .'\   /`.
%         .'.-.`-'.-.`.
%    ..._:   .-. .-.   :_...
%  .'    '-.(o ) (o ).-'    `.
% :  _    _ _`~(_)~`_ _    _  :
%:  /:   ' .-=_   _=-. `   ;\  :
%:   :|-.._  '     `  _..-|:   :
% :   `:| |`:-:-.-:-:'| |:'   :
%  `.   `.| | | | | | |.'   .'
%    `.   `-:_| | |_:-'   .'
%      `-._   ````    _.-'
%          ``-------''
\begin{abstract}
Job monitoring on high-performance computing clusters is important for evaluating hardware performance, troubleshooting failed jobs, identifying inefficient jobs and more. The combination of the Prometheus monitoring framework and the Grafana visualization toolkit has proven successful in recent years. This work shows how four Prometheus exporters can be configured for a Slurm cluster to provide detailed job-level information on CPU/GPU efficiencies and CPU/GPU memory usage as well as node-level Network File System (NFS) statistics and cluster-level General Parallel File System (GPFS) activity. A novel approach was devised to efficiently store a summary of this data in the Slurm database for each completed job. The open-source job monitoring platform introduced here can be used for batch, interactive and Open OnDemand jobs. Several tools are presented that use the Prometheus and Slurm databases to create dashboards, utilization reports and alerts.
\end{abstract}

%%
%% The code below is generated by the tool at http://dl.acm.org/ccs.cfm.
%% Please copy and paste the code instead of the example below.
%%
\begin{CCSXML}
<ccs2012>
   <concept>
       <concept_id>10002951.10003227.10010926</concept_id>
       <concept_desc>Information systems~Computing platforms</concept_desc>
       <concept_significance>500</concept_significance>
       </concept>
 </ccs2012>
\end{CCSXML}

\ccsdesc[500]{Information systems~Computing platforms}


%%
%% Keywords. The author(s) should pick words that accurately describe
%% the work being presented. Separate the keywords with commas.
\keywords{Job Monitoring, Slurm, Prometheus, Grafana, GPUs, Alerts}

%% A "teaser" image appears between the author and affiliation
%% information and the body of the document, and typically spans the
%% page.


\received{3 March 2023}
\received[revised]{X XXXXX 2023}
\received[accepted]{X XXXXX 2023}

%%
%% This command processes the author and affiliation and title
%% information and builds the first part of the formatted document.
\maketitle

\input{JOBINTRO.tex}
\input{JOBDESIGN.tex}
\input{JOBTOOLS.tex}

\section{Summary}
The Jobstats platform is built on Prometheus, Grafana and Slurm. The speed and efficiency of the time-series database provided by Prometheus is central to our design. The multiple exporters produce a rich dataset which is harnessed by the external tools such as \texttt{jobstats}. A standard server is sufficient to support Prometheus. The Jobstats platform has proven successful at an institution with 100,000 CPU-cores and 500 GPUs with an exporter sampling period of 30 seconds. While setting up the platform requires touching numerous files, the procedure for doing so is well-documented and the benefits of Jobstats over alternative job monitoring platforms are many. The Jobstats platform is particularly relevant to institutions with GPU clusters. The custom notes that appear at the bottom of the \texttt{jobstats} output have proven to be very useful in guiding users. There are plans to extend the platform to provide information on data storage.

Can the platform be configured for job schedulers other than Slurm such as PBS? The most difficult piece to adjust is likely to be the cgroups based process accounting. Other exporters should be easier to modify for other schedulers.
Presumably there is something analogous to using the \texttt{AdminComment} field for the target scheduler.

Please direct all questions concerning this work to Princeton Research Computing at \href{mailto:cses@princeton.edu}{cses@princeton.edu}.

%\begin{figure}
%  \centering
%  \includegraphics[width=\linewidth]{sample-franklin}
%  \caption{1907 Franklin Model D roadster. Photograph by Harris \&
%    Ewing, Inc. [Public domain], via Wikimedia
%    Commons. (\url{https://goo.gl/VLCRBB}).}
%  \Description{A woman and a girl in white dresses sit in an open car.}
%\end{figure}

%%
%% The acknowledgments section is defined using the "acks" environment
%% (and NOT an unnumbered section). This ensures the proper
%% identification of the section in the article metadata, and the
%% consistent spelling of the heading.
\begin{acks}
Discussions with William Wichser helped to shape the design of the Jobstats platform. The authors are grateful to Galen Collier and Asya Dvorkin for reviewing this work. Carolina Roe-Raymond and Kevin Abbey contributed to the custom notes generated by the \texttt{jobstats} command.
\end{acks}

%%
%% The next two lines define the bibliography style to be used, and
%% the bibliography file.
\bibliographystyle{ACM-Reference-Format}
\bibliography{JOBSTATS}

%%
%% If your work has an appendix, this is the place to put it.
\input{JOBAPPENDIX.tex}

\end{document}
\endinput
%%
%% End of file `sample-acmsmall-conf.tex'.
